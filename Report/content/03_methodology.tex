%%%%%%%%%%%%%%%%%%%%%%%%%%%%%%%%%%%%%%%%%%%%%%%%%%%%%%%%%%%%%%%%%%%%%%%
%%%%%%%%%%%%%%%%%%%%%%%%%%%%%%%%%%%%%%%%%%%%%%%%%%%%%%%%%%%%%%%%%%%%%%%
%%%%%                                                                 %
%%%%%     03_methodology.tex                                          %
%%%%%                                                                 %
%%%%% Author:      Miguel Correa, Elio Warner                         %
%%%%% Created:     22.03.2024                                         %
%%%%% Description: - Design choices for WDT                           %
%%%%%                (e.g., countdown vs. count-up, reset duration)   %
%%%%%              - Testing strategy (testbenches, simulation tools) %
%%%%%              - Debugging and iteration process                  %
%%%%%                                                                 %
%%%%%%%%%%%%%%%%%%%%%%%%%%%%%%%%%%%%%%%%%%%%%%%%%%%%%%%%%%%%%%%%%%%%%%%
%%%%%%%%%%%%%%%%%%%%%%%%%%%%%%%%%%%%%%%%%%%%%%%%%%%%%%%%%%%%%%%%%%%%%%%


\chapter{Methodology}
\label{chap:methodology}

\section{Design choices for WDT (e.g., countdown vs. count-up, reset duration)}
Explain that we decided to make a simple watchdog, independent from croc to keep it simple.
Then, we added a wrapper to be able to connect it to OBI and croc.
Our watchdog uses cout-up (was our own choice) and have a 1 cycle reset duration, 
wich we found enough.

\section{Testing strategy (testbenches, simulation tools)}
At each step we tested our watchdog, from the simple watchdog to the communication with obi and finally the whole connected croc.
If we have time, we want to test a real life scenario (maybe a gpio that returns nothing so our program is stuck?).

\section{Debugging and iteration process}
By separating the work in easy, well separated steps it was easy to debbug.

