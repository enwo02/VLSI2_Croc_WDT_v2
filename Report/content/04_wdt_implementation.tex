%%%%%%%%%%%%%%%%%%%%%%%%%%%%%%%%%%%%%%%%%%%%%%%%%%%%%%%%%%%%%%%%%%%%%%%
%%%%%%%%%%%%%%%%%%%%%%%%%%%%%%%%%%%%%%%%%%%%%%%%%%%%%%%%%%%%%%%%%%%%%%%
%%%%%                                                                 %
%%%%%     04_wdt_implementation.tex                                   %
%%%%%                                                                 %
%%%%% Author:      Miguel Correa, Elio Wanner                         %
%%%%% Created:     17.03.2025                                         %
%%%%% Description: In here is the description of the hardware         %
%%%%%                                                                 %
%%%%%%%%%%%%%%%%%%%%%%%%%%%%%%%%%%%%%%%%%%%%%%%%%%%%%%%%%%%%%%%%%%%%%%%
%%%%%%%%%%%%%%%%%%%%%%%%%%%%%%%%%%%%%%%%%%%%%%%%%%%%%%%%%%%%%%%%%%%%%%%

\chapter{Hardware Architecture}
This chapter describes the hardware architecture of the Watchdog Timer that we implemented in the Croc SoC.
In the different sections we talk about different implementations with different features and complexities.

\section{Simple Watchdog Timer}
\subsection{Hardware implementation}
The first implementation of the Watchdog Timer is the simplest one we could imagine. 
As shown in Figure \ref{fig:simple_WDT}, the WDT is composed of
very few components. The main component is the counter, which is
incremented every clock cycle and implemented as a flip-flop. The counter is connected to a comparator
that checks if the counter has reached a certain value. If the counter
reaches the value, the comparator will trigger a reset signal that will
reset the system. Otherwise, we keep incrementing the counter. As soon as we get
a kick signal, the counter is reset to zero.

\begin{figure}[h]
\centering
\includegraphics[width=0.7\textwidth]{./figures/simple_WDT}
\caption{Simple Watchdog Timer Architecture}
\label{fig:simple_WDT}
\end{figure}

\subsection{Tests and Results}

\subsection{Conclusion}

\section{Watchdog Timer with 2 stages}
This is for later.